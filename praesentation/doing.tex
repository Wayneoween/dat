\begin{frame}{Projekt - Idee}
  \begin{center}
    \includegraphics[width=0.9\textwidth]{images/grobarchitektur}
  \end{center}
\end{frame}

\section{Was wir dann gemacht haben}

\begin{frame}{Projekt - Realität}
  \begin{center}
    \includegraphics[width=0.9\textwidth]{images/realworld}
  \end{center}
\end{frame}

\begin{frame}{deCONZ - REST-API - GET 1/2}
  \Large
  GET-Requests: Informationsbeschaffung
  \center{
    \begin{tabular}{ l l }
      \texttt{<url>/<apikey>} & \\
      Lichter & \texttt /\alert{\texttt{lights}}\\
      Gruppen & \texttt /\alert{\texttt{groups}}\\
      Szenen & \hspace{1em}$\looparrowdownright$\hspace{1em}
        \texttt{/<id>/\alert{\texttt{scenes}}}\\
    \end{tabular}
  }
  \flushleft
  für \alert{alle} Lichter, Gruppen und Szenen.
\end{frame}

\begin{frame}{deCONZ - REST-API - GET 2/2}
  \Large
  GET-Requests: Informationsbeschaffung
  \center{
    \begin{tabular}{ l l }
      \texttt{<url>/<apikey>} & \\
      Lichter & \texttt /\alert{\texttt{lights/<id>}}\\
      Gruppen & \texttt /\alert{\texttt{groups/<id>}}\\
      Szenen & \hspace{1em}$\looparrowdownright$\hspace{1em}
        \texttt /\alert{\texttt{scenes/<id>}}\\
    \end{tabular}
  }
  \flushleft
  für \alert{einzelne} Lichter, Gruppen und Szenen
\end{frame}

\begin{frame}{deCONZ - REST-API - PUT}
  \Large
  PUT-Requests: Datenaktualisierung
  \center{
    \begin{tabular}{ l l }
      \texttt{<url>/<apikey>} & \\
      Lichter & \texttt /\alert{\texttt{lights/<id>/state}}\\
      Gruppen & \texttt /\alert{\texttt{groups/<id>}}\\
      Szenen & \hspace{1em}$\looparrowdownright$\hspace{1em}
        \texttt /\alert{\texttt{scenes/<id>}}\\
    \end{tabular}
  }
  \flushleft
  für \alert{einzelne} Lichter, Gruppen und Szenen
\end{frame}

\begin{frame}{deCONZ - REST-API - POST}
  \Large
  POST-Requests: Erstellung
  \center{
    \begin{tabular}{ l l }
      \texttt{<url>/<apikey>} & \\
      Gruppen & \texttt /\alert{\texttt{groups}}\\
      Szenen & \hspace{1em}$\looparrowdownright$\hspace{1em}
        \texttt{/<id>/\alert{\texttt{scenes}}}\\
    \end{tabular}
  }
  \flushleft
  von Gruppen und Szenen
\end{frame}

\begin{frame}{deCONZ - REST-API - DELETE}
  \Large
  DELETE-Requests: Löschung
  \center{
    \begin{tabular}{ l l }
      \texttt{<url>/<apikey>} & \\
      Gruppen & \texttt /\alert{\texttt{groups}}\\
      Szenen & \hspace{1em}$\looparrowdownright$\hspace{1em}
        \texttt{/<id>/\alert{\texttt{scenes}}}\\
    \end{tabular}
  }
  \flushleft
  von Gruppen und Szenen
\end{frame}

\begin{frame}{}
  \begin{center}
    \vspace{-0.2cm}
    \makebox[\textwidth][c]{\includegraphics[width=1.25\paperwidth]{images/matrix}}
    \label{fig:matrix}
  \end{center}
\end{frame}

\begin{frame}{lolo}
  \Large
  \begin{quote}
    Wir nennen das einfach \alert{lolo},\\
    das steht dann für ``\textbf Lights \textbf On, \textbf Lights \textbf Off''
    und fertig.
    \begin{flushright}
      \small
      -- \texttt{lolo}-Entwicklerteam
    \end{flushright}
  \end{quote}
\end{frame}

\begin{frame}{lolo - Entwicklung}
  \Large
  \center{\includegraphics[width=0.3\linewidth]{images/ruby}}
  Ruby
  \begin{itemize}
    \item Objektorientiert
    \item Interpretiert
    \item Viele Module verfügbar
  \end{itemize}
\end{frame}

\begin{frame}{lolo - Entwicklung - Module}
  \Large
  Wir nutzen hauptsächlich folgende RubyGems:
  \center{
    \begin{tabular}{ l l }
      \texttt{clamp} & Parsen von CLI-Optionen\\
      \texttt{json} & Parsen von JSON-Objekten\\
      \texttt{yaml} & Caching von JSON-Objekten\\
      \texttt{rest-client} & Highlevel REST-Abstraktion\\
    \end{tabular}
  }
\end{frame}