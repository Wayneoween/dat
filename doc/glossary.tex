\newglossaryentry{gls-rpi}
{
  name={RaspBerry Pi},
  description={Kleiner Einplatinencomputer im Kreditkartenformat, entwickelt von
    der \emph{Raspberry Pi Foundation}.	Basiert auf einem ein Ein-Chip-System von Broadcom
    mit einem ARM-Mikroprozessor und kostet zwischen 25 und 35\$.},
}
\newglossaryentry{gls-zb}
{
  name={ZigBee},
  description={Spezifikation für drahtlose Netzwerke mit geringem Datenaufkommen,
    wie z. B. Hausautomation, Sensornetzwerke, Lichttechnik. Der Schwerpunkt von
    ZigBee liegt in kurzreichweitigen Netzwerken (10 bis 100 Meter).
    \newline \url{http://www.zigbee.org}}
}
\newglossaryentry{gls-gatt}
{
  name={Generic Attribute Profile},
  description={\gls{gls-bt-profil} zur ernergieeffizienten Übertragung von Sensordaten und kleiner
    Datenmengen.},
}
\newglossaryentry{gls-bt-profil}
{
  name={Bluetooth-Profil},
  description={Schnittstellenspezifikation der Bluetooth Special Interest Group für die drahtlose
    Kommunikation in einer Bluetooth-Umgebung.},
}
\newglossaryentry{gls-apt}
{
  name={Advanced Packaging Tool},
  description={Paketverwaltungssystem aus dem Bereich von Debian/Ubuntu mit dem Ziel
    eine einfache Möglichkeit zur Suche, Installation und Aktualisierung von Programmpaketen
    zur Verfügung zu stellen.},
}
\newglossaryentry{gls-rb}
{
  name={Ruby},
  description={Einfach zu lesende, objektorierte, höhere Programmiersprache deren Programme
    zur Laufzeit interpretiert werden.},
}

\newacronym{apt}{APT}{\gls{gls-apt}}
\newacronym{rpi}{RPi}{\gls{gls-rpi}}
\newacronym{gatt}{GATT}{\gls{gls-gatt}}
\newacronym{gpio}{GPIO}{General Purpose Input/Output}
\newacronym{uart}{UART}{Universal Asynchronous Receiver Transmitter}
\newacronym{deconz}{deCONZ}{dresden elektronik CONtrol Zigbee
  \newline \url{https://www.dresden-elektronik.de/funktechnik/products/software/pc/deconz/}}
\newacronym{noobs}{NOOBS}{New Out Of the Box Software}